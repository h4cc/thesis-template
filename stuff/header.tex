%%%%%%%%%%%%%%%%%%%%%%%%%%%%%%%%%%%%%%%%%%%%%%%%%%%%%%%%%%%%%%%%%%%%%%%
\documentclass
  [ %twoside,            % beidseitiger Druck
   BCOR=10mm          % Bindekorrektur
  , openright          % Kapitel beginnen auf einer rechten Seite
  , listof=totoc       % Verzeichnisse im Inhaltsverzeichnis
  , bibliography=totoc % Literaturverzeichnis im Inhaltsverzeichnis
  , parskip=half       % Absätze durch einen vergrößerten Zeilenabstand getrennt
  %, draft              % Entwurfsversion
  ]{scrreprt}          % Dokumentenklasse: KOMA-Script Buch


%%%%%%%%%%%%%%%%%%%%%%%%%%%%%%%%%%%%%%%%%%%%%%%%%%%%%%%%%%%%%%%%%%%%%%%
% Packages
%%%%%%%%%%%%%%%%%%%%%%%%%%%%%%%%%%%%%%%%%%%%%%%%%%%%%%%%%%%%%%%%%%%%%%%
\usepackage{scrhack}


\usepackage{ifpdf}
\ifpdf
  \usepackage{ae}               % Fonts für pdfLaTeX, falls keine cm-super-Fonts installiert
  \usepackage{microtype}        % optischer Randausgleich, falls pdflatex verwandt
  \usepackage[pdftex]{graphicx} % Grafiken in pdfLaTeX
\else
  \usepackage[dvips]{graphicx}  % Grafiken und normales LaTeX
\fi

\usepackage[utf8]{inputenc}         % Input encoding (allow direct use of special characters like "ä")
%\usepackage[english]{babel}
\usepackage[ngerman]{babel}
\usepackage[T1]{fontenc}
\usepackage[automark]{scrpage2}     % Schickerer Satzspiegel mit KOMA-Script
\usepackage{setspace}               % Allow the modification of the space between lines
\usepackage{booktabs}               % Netteres Tabellenlayout
\usepackage{multicol}               % Mehrspaltige Bereiche
\usepackage{quotchap}               % Beautiful chapter decoration
\usepackage[printonlyused]{acronym} % list of acronyms and abbreviations
\usepackage{subfig}                 % allow sub figures
\usepackage{tabularx}              % Tabellen mit fester Breite
\usepackage{url}                   % URLs als Links darstellen
\usepackage{wrapfig}               % Wrappen von Bildern usw.
\usepackage{fancyvrb}


% Literaturverzeichniss
%\usepackage{babelbib} %Erzeugt deutsches Literaturverzeichnis
%\usepackage[notocbib,nosectionbib]{apacite}
%\usepackage{natbib}
\usepackage[sectionbib]{natbib}


% Layout
\pagestyle{scrheadings}
%\pagestyle{empty}
\clubpenalty = 20000
\widowpenalty = 20000
\displaywidowpenalty = 20000
%\linespread {1.25}

\makeatletter
\renewcommand{\fps@figure}{htbp}
\makeatother

\addto\captionsngerman{\renewcommand{\figurename}{Abb.}}

%% Document properties %%%%%%%%%%%%%%%%%%%%%%%%%%%%%%%%%%%%%%%%%%%%%%%%
\newcommand{\projname}{Titel der Thesis}
\newcommand{\titel}{\projname}
\newcommand{\untertitel}{Untertitel der Thesis}
\newcommand{\authorname}{Deine Name}
\newcommand{\thesisname}{Bachelor Thesis}
\newcommand{\Datum}{30. April 2013}

\ifpdf
  \usepackage{hyperref}
  \definecolor{darkblue}{rgb}{0,0,.5}
  \hypersetup
    { colorlinks%=false
    , breaklinks=true
    , linkcolor=darkblue
    , menucolor=darkblue
    , urlcolor=darkblue
    , pdftitle={\projname -- \untertitel}
    , pdfsubject={\thesisname}
    , pdfauthor={\authorname}
    }
\else
\fi

%% Listings %%%%%%%%%%%%%%%%%%%%%%%%%%%%%%%%%%%%%%%%%%%%%%%%%%%%%%%%%
\usepackage{listings}
\KOMAoptions{listof=totoc} % necessary because of scrhack
\renewcommand{\lstlistlistingname}{List of Listings}
\lstset
  { basicstyle=\small\ttfamily
  , breaklines=true
  , captionpos=b
  , showstringspaces=false
  , keywordstyle={}
  }

\lstnewenvironment{inlinehaskell}
{\spacing{1}\lstset{language=haskell,nolol,aboveskip=\bigskipamount}}
{\endspacing}

\lstnewenvironment{inlinexml}
{\spacing{1}\lstset{language=XML,nolol,aboveskip=\bigskipamount}}
{\endspacing}

\newcommand{\haskellinput}[2][]{
  \begin{spacing}{1}
  \lstinputlisting[language=Haskell,nolol,aboveskip=\bigskipamount,#1]{#2}
  \end{spacing}
}

\newcommand{\haskellcode}[2][]{\mylisting[#1,language=Haskell]{#2}}

\newcommand{\mylisting}[2][]{
\begin{spacing}{1}
\lstinputlisting[frame=lines,aboveskip=2\bigskipamount,#1]{#2}
\end{spacing}
}

